\documentclass[11pt]{article}
\usepackage[utf8]{inputenc}
\usepackage[T1]{fontenc}
\usepackage[margin=1in]{geometry}
\usepackage{booktabs}
\usepackage{longtable}
\usepackage{array}
\usepackage{xcolor}
\usepackage{hyperref}
\usepackage{fancyhdr}

\definecolor{okgreen}{RGB}{40,167,69}
\definecolor{errorred}{RGB}{220,53,69}

\pagestyle{fancy}
\fancyhf{}
\rhead{Citation Verification Report}
\lhead{GenAI-RAG-EEG Paper}
\rfoot{Page \thepage}

\title{\textbf{Citation and Reference Verification Report}\\[0.5em]
\large GenAI-RAG-EEG: A Novel Hybrid Deep Learning Architecture\\with Retrieval-Augmented Generation for Explainable\\EEG-Based Stress and Cognitive Workload Classification}
\author{Verification Date: \today}
\date{}

\begin{document}
\maketitle

\section*{Verification Summary}
\begin{center}
\renewcommand{\arraystretch}{1.3}
\begin{tabular}{|l|c|}
\hline
\textbf{Metric} & \textbf{Value} \\
\hline
Total Citations in Paper & 40 \\
Total Bibliography Entries & 40 \\
Missing References & \textcolor{okgreen}{\textbf{0}} \\
Orphan References (Uncited) & \textcolor{okgreen}{\textbf{0}} \\
Verification Status & \textcolor{okgreen}{\textbf{PASSED}} \\
\hline
\end{tabular}
\end{center}

\vspace{1em}
\noindent\textbf{Result:} All citations are correctly linked to their corresponding bibliography entries.

\section*{Detailed Citation-Reference Mapping}

\begin{longtable}{|c|p{4cm}|p{2.5cm}|c|c|}
\hline
\textbf{\#} & \textbf{Citation Key} & \textbf{Cited at Lines} & \textbf{Bib Line} & \textbf{Status} \\
\hline
\endfirsthead
\hline
\textbf{\#} & \textbf{Citation Key} & \textbf{Cited at Lines} & \textbf{Bib Line} & \textbf{Status} \\
\hline
\endhead
\hline
\endfoot

1 & who2023stress & 115 & 1221 & \textcolor{okgreen}{OK} \\
2 & hassard2018cost & 115 & 1224 & \textcolor{okgreen}{OK} \\
3 & cohen1983pss & 115 & 1227 & \textcolor{okgreen}{OK} \\
4 & lovibond1995dass & 115 & 1230 & \textcolor{okgreen}{OK} \\
5 & teplan2002fundamentals & 119 & 1233 & \textcolor{okgreen}{OK} \\
6 & mcewen2007physiology & 119 & 1236 & \textcolor{okgreen}{OK} \\
7 & klimesch1999alpha & 119 & 1239 & \textcolor{okgreen}{OK} \\
8 & ray1985eeg & 119 & 1242 & \textcolor{okgreen}{OK} \\
9 & harmony2009eeg & 119 & 1245 & \textcolor{okgreen}{OK} \\
10 & davidson2004well & 119 & 1248 & \textcolor{okgreen}{OK} \\
11 & subhani2017machine & 123, 911 & 1251 & \textcolor{okgreen}{OK} \\
12 & sharma2012objective & 123, 912 & 1254 & \textcolor{okgreen}{OK} \\
13 & healey2005detecting & 123 & 1257 & \textcolor{okgreen}{OK} \\
14 & alshargie2016mental & 125 & 1260 & \textcolor{okgreen}{OK} \\
15 & arsalan2019classification & 125 & 1263 & \textcolor{okgreen}{OK} \\
16 & hou2015eeg & 125 & 1266 & \textcolor{okgreen}{OK} \\
17 & alhagry2017emotion & 129, 916 & 1269 & \textcolor{okgreen}{OK} \\
18 & schirrmeister2017deep & 129 & 1272 & \textcolor{okgreen}{OK} \\
19 & lawhern2018eegnet & 131, 918 & 1275 & \textcolor{okgreen}{OK} \\
20 & li2019hierarchical & 131 & 1278 & \textcolor{okgreen}{OK} \\
21 & tripathi2017using & 131, 915 & 1281 & \textcolor{okgreen}{OK} \\
22 & chen2021accurate & 133, 917 & 1284 & \textcolor{okgreen}{OK} \\
23 & zhang2020spatial & 133 & 1287 & \textcolor{okgreen}{OK} \\
24 & song2020eeg & 133, 919 & 1290 & \textcolor{okgreen}{OK} \\
25 & tao2020eeg & 137 & 1293 & \textcolor{okgreen}{OK} \\
26 & wang2022transformers & 137 & 1296 & \textcolor{okgreen}{OK} \\
27 & li2023bihemisphere & 137 & 1299 & \textcolor{okgreen}{OK} \\
28 & gonzalez2024deep & 139 & 1302 & \textcolor{okgreen}{OK} \\
29 & hwang2020learning & 139 & 1305 & \textcolor{okgreen}{OK} \\
30 & tonekaboni2019clinicians & 143 & 1308 & \textcolor{okgreen}{OK} \\
31 & holzinger2019causability & 143 & 1311 & \textcolor{okgreen}{OK} \\
32 & cui2020eeg & 143 & 1314 & \textcolor{okgreen}{OK} \\
33 & lewis2020rag & 145 & 1317 & \textcolor{okgreen}{OK} \\
34 & zhang2024medical & 145 & 1320 & \textcolor{okgreen}{OK} \\
35 & rudin2019stop & 152 & 1323 & \textcolor{okgreen}{OK} \\
36 & schmidt2018eegmat & 154 & 1326 & \textcolor{okgreen}{OK} \\
37 & lotte2018review & 156 & 1329 & \textcolor{okgreen}{OK} \\
38 & gal2016dropout & 158 & 1332 & \textcolor{okgreen}{OK} \\
39 & wang2023evidence & 160 & 1335 & \textcolor{okgreen}{OK} \\
40 & koelstra2012deap & 193 & 1338 & \textcolor{okgreen}{OK} \\
\hline
\end{longtable}

\section*{Full Reference Details}

\begin{longtable}{|c|p{3.5cm}|p{9cm}|}
\hline
\textbf{\#} & \textbf{Key} & \textbf{Full Reference} \\
\hline
\endfirsthead
\hline
\textbf{\#} & \textbf{Key} & \textbf{Full Reference} \\
\hline
\endhead
\hline
\endfoot

1 & who2023stress & World Health Organization, ``Mental health: Strengthening our response,'' WHO Fact Sheet, 2023. \\
2 & hassard2018cost & J. Hassard et al., ``The cost of work-related stress to society: A systematic review,'' J. Occup. Health Psychol., vol. 23, no. 1, pp. 1--17, 2018. \\
3 & cohen1983pss & S. Cohen, T. Kamarck, and R. Mermelstein, ``A global measure of perceived stress,'' J. Health Soc. Behav., vol. 24, pp. 385--396, 1983. \\
4 & lovibond1995dass & S. H. Lovibond and P. F. Lovibond, ``Manual for the Depression Anxiety Stress Scales,'' Psychology Foundation, Sydney, 1995. \\
5 & teplan2002fundamentals & M. Teplan, ``Fundamentals of EEG measurement,'' Meas. Sci. Rev., vol. 2, no. 2, pp. 1--11, 2002. \\
6 & mcewen2007physiology & B. S. McEwen, ``Physiology and neurobiology of stress and adaptation,'' Physiol. Rev., vol. 87, no. 3, pp. 873--904, 2007. \\
7 & klimesch1999alpha & W. Klimesch, ``EEG alpha and theta oscillations reflect cognitive and memory performance,'' Brain Res. Rev., vol. 29, pp. 169--195, 1999. \\
8 & ray1985eeg & W. J. Ray and H. W. Cole, ``EEG alpha activity reflects attentional demands,'' Science, vol. 228, pp. 750--752, 1985. \\
9 & harmony2009eeg & T. Harmony, ``The functional significance of delta oscillations in cognitive processing,'' Front. Integr. Neurosci., vol. 7, p. 83, 2009. \\
10 & davidson2004well & R. J. Davidson, ``Well-being and affective style: Neural substrates and biobehavioural correlates,'' Phil. Trans. R. Soc. B, vol. 359, pp. 1395--1411, 2004. \\
11 & subhani2017machine & A. R. Subhani et al., ``Machine learning framework for the detection of mental stress at multiple levels,'' IEEE Access, vol. 5, pp. 13545--13556, 2017. \\
12 & sharma2012objective & N. Sharma and T. Gedeon, ``Objective measures, sensors and computational techniques for stress recognition,'' Comput. Methods Programs Biomed., vol. 108, pp. 1287--1301, 2012. \\
13 & healey2005detecting & J. A. Healey and R. W. Picard, ``Detecting stress during real-world driving tasks using physiological sensors,'' IEEE Trans. Intell. Transp. Syst., vol. 6, no. 2, pp. 156--166, 2005. \\
14 & alshargie2016mental & F. Al-shargie et al., ``Mental stress assessment using simultaneous measurement of EEG and fNIRS,'' Biomed. Opt. Express, vol. 7, no. 10, pp. 3882--3898, 2016. \\
15 & arsalan2019classification & A. Arsalan et al., ``Classification of perceived mental stress using a commercially available EEG headband,'' IEEE J. Biomed. Health Inform., vol. 23, no. 6, pp. 2257--2264, 2019. \\
16 & hou2015eeg & X. Hou et al., ``EEG based stress monitoring,'' IEEE Int. Conf. Syst. Man Cybern., pp. 3110--3115, 2015. \\
17 & alhagry2017emotion & S. Alhagry, A. A. Fahmy, and R. A. El-Khoribi, ``Emotion recognition based on EEG using LSTM recurrent neural network,'' Int. J. Adv. Comput. Sci. Appl., vol. 8, no. 10, pp. 355--358, 2017. \\
18 & schirrmeister2017deep & R. T. Schirrmeister et al., ``Deep learning with convolutional neural networks for EEG decoding and visualization,'' Hum. Brain Mapp., vol. 38, no. 11, pp. 5391--5420, 2017. \\
19 & lawhern2018eegnet & V. J. Lawhern et al., ``EEGNet: A compact convolutional neural network for EEG-based brain-computer interfaces,'' J. Neural Eng., vol. 15, no. 5, p. 056013, 2018. \\
20 & li2019hierarchical & Y. Li et al., ``A bi-hemisphere domain adversarial neural network model for EEG emotion recognition,'' IEEE Trans. Affect. Comput., vol. 12, no. 2, pp. 494--504, 2019. \\
21 & tripathi2017using & S. Tripathi et al., ``Using deep and convolutional neural networks for accurate emotion classification on DEAP dataset,'' Proc. AAAI Conf. Artif. Intell., pp. 4746--4752, 2017. \\
22 & chen2021accurate & J. Chen et al., ``Accurate EEG-based emotion recognition on combined CNN-LSTM with attention mechanism,'' Neural Networks, vol. 143, pp. 485--496, 2021. \\
23 & zhang2020spatial & T. Zhang et al., ``Spatial-temporal recurrent neural network for emotion recognition,'' IEEE Trans. Cybern., vol. 49, no. 3, pp. 839--847, 2020. \\
24 & song2020eeg & T. Song et al., ``EEG emotion recognition using dynamical graph convolutional neural networks,'' IEEE Trans. Affect. Comput., vol. 11, no. 3, pp. 532--541, 2020. \\
25 & tao2020eeg & W. Tao et al., ``EEG-based emotion recognition via channel-wise attention and self attention,'' IEEE Trans. Affect. Comput., 2020. \\
26 & wang2022transformers & Z. Wang et al., ``Transformers for EEG-based emotion recognition: A hierarchical spatial information learning model,'' IEEE Sens. J., vol. 22, pp. 4359--4368, 2022. \\
27 & li2023bihemisphere & Y. Li et al., ``Bi-hemisphere discrepancy for cross-session EEG emotion recognition,'' IEEE Trans. Affect. Comput., vol. 14, pp. 1068--1080, 2023. \\
28 & gonzalez2024deep & H. Gonzalez et al., ``Deep learning for EEG-based stress detection: A comprehensive benchmark,'' IEEE Trans. Neural Syst. Rehabil. Eng., 2024. \\
29 & hwang2020learning & S. Hwang et al., ``Learning subject-independent representation for EEG-based drowsiness detection,'' IEEE Access, vol. 8, pp. 86736--86746, 2020. \\
30 & tonekaboni2019clinicians & S. Tonekaboni et al., ``What clinicians want: Contextualizing explainable machine learning for clinical end use,'' Proc. Mach. Learn. Healthc. Conf., pp. 359--380, 2019. \\
31 & holzinger2019causability & A. Holzinger et al., ``Causability and explainability of artificial intelligence in medicine,'' WIREs Data Min. Knowl. Discov., vol. 9, no. 4, e1312, 2019. \\
32 & cui2020eeg & H. Cui et al., ``EEG-based emotion recognition: A review of recent progress,'' IEEE Trans. Cogn. Dev. Syst., vol. 12, no. 2, pp. 217--231, 2020. \\
33 & lewis2020rag & P. Lewis et al., ``Retrieval-augmented generation for knowledge-intensive NLP tasks,'' Proc. NeurIPS, vol. 33, pp. 9459--9474, 2020. \\
34 & zhang2024medical & S. Zhang et al., ``Medical RAG: Retrieval-augmented generation for clinical decision support,'' Nature Digit. Med., 2024. \\
35 & rudin2019stop & C. Rudin, ``Stop explaining black box machine learning models for high stakes decisions and use interpretable models instead,'' Nature Mach. Intell., vol. 1, pp. 206--215, 2019. \\
36 & schmidt2018eegmat & P. Schmidt et al., ``Introducing EEGMAT, a multimodal dataset for wearable stress and affect detection,'' Proc. ICMI, pp. 400--408, 2018. \\
37 & lotte2018review & F. Lotte et al., ``A review of classification algorithms for EEG-based brain-computer interfaces,'' J. Neural Eng., vol. 15, no. 3, p. 031005, 2018. \\
38 & gal2016dropout & Y. Gal and Z. Ghahramani, ``Dropout as a Bayesian approximation: Representing model uncertainty in deep learning,'' Proc. ICML, pp. 1050--1059, 2016. \\
39 & wang2023evidence & S. Wang et al., ``Evidence-grounded neural network explanations for healthcare,'' Nature Comput. Sci., 2023. \\
40 & koelstra2012deap & S. Koelstra et al., ``DEAP: A database for emotion analysis using physiological signals,'' IEEE Trans. Affect. Comput., vol. 3, no. 1, pp. 18--31, 2012. \\
\hline
\end{longtable}

\section*{Verification Script}
The following Python script can be used to re-verify citations at any time:

\begin{verbatim}
import re
with open('eeg-stress-rag.tex', 'r') as f:
    content = f.read()
    lines = content.split('\n')

citations = set()
bibitems = set()

for line in lines:
    for m in re.findall(r'\\cite\{([^}]+)\}', line):
        citations.update(m.split(','))
    m = re.match(r'\\bibitem\{([^}]+)\}', line)
    if m: bibitems.add(m.group(1))

missing = citations - bibitems
orphan = bibitems - citations
print(f"Missing refs: {missing or 'None'}")
print(f"Orphan refs: {orphan or 'None'}")
\end{verbatim}

\end{document}
